%%%%%%%%%%%%%%%%%%%%%%%%%%%%%%%%%%%%%%%%%
% "ModernCV" CV and Cover Letter
% LaTeX Template
% Version 1.3 (29/10/16)
%
% This template has been downloaded from:
% http://www.LaTeXTemplates.com
%
% Original author:
% Xavier Danaux (xdanaux@gmail.com) with modifications by:
% Vel (vel@latextemplates.com)
%
% License:
% CC BY-NC-SA 3.0 (http://creativecommons.org/licenses/by-nc-sa/3.0/)
%
% Important note:
% This template requires the moderncv.cls and .sty files to be in the same 
% directory as this .tex file. These files provide the resume style and themes 
% used for structuring the document.
%
%%%%%%%%%%%%%%%%%%%%%%%%%%%%%%%%%%%%%%%%%

%----------------------------------------------------------------------------------------
%	PACKAGES AND OTHER DOCUMENT CONFIGURATIONS
%----------------------------------------------------------------------------------------

\documentclass[11pt,a4paper,sans]{moderncv} % Font sizes: 10, 11, or 12; paper sizes: a4paper, letterpaper, a5paper, legalpaper, executivepaper or landscape; font families: sans or roman

\moderncvstyle{classic} % CV theme - options include: 'casual' (default), 'classic', 'oldstyle' and 'banking'
\moderncvcolor{blue} % CV color - options include: 'blue' (default), 'orange', 'green', 'red', 'purple', 'grey' and 'black'

\usepackage{lipsum} % Used for inserting dummy 'Lorem ipsum' text into the template

\usepackage[scale=0.75]{geometry} % Reduce document margins
%\setlength{\hintscolumnwidth}{3cm} % Uncomment to change the width of the dates column
%\setlength{\makecvtitlenamewidth}{10cm} % For the 'classic' style, uncomment to adjust the width of the space allocated to your name

%----------------------------------------------------------------------------------------
%	NAME AND CONTACT INFORMATION SECTION
%----------------------------------------------------------------------------------------

\firstname{Matteo} % Your first name
\familyname{Ansaldi} % Your last name

% All information in this block is optional, comment out any lines you don't need
\title{Curriculum Vitae}
\address{Via Borgo San Dalmazzo, 6}{12012, Boves (CN), Italy}
\mobile{+39 366 351 6370}
\email{ansaldimatteo@outlook.com}
\extrainfo{www.linkedin.com/in/ansaldi-matteo}

%----------------------------------------------------------------------------------------

\begin{document}


%----------------------------------------------------------------------------------------
%	CURRICULUM VITAE
%----------------------------------------------------------------------------------------

\makecvtitle % Print the CV title

%----------------------------------------------------------------------------------------
%	EDUCATION SECTION
%----------------------------------------------------------------------------------------

\section{Education}

\cventry{2017--2019}{Masters Degree in Computer Engineering}{Politecnico di Torino}{Turin}{\textit{110/110}}{}  % Arguments not required can be left empty
\cventry{2013--2017}{Bachelors Degree in Computer Engineering}{Politecnico di Torino}{Turin}{\textit{92/110}}{}

\section{Masters Thesis}

\cvitem{Title}{\emph{Camera-less context detection in mobile platforms}}
\cvitem{Supervisors}{Professor M. Poncino \& Associate Professor D. Jahier Pagliari}
\cvitem{Description}{This thesis studied the potential to use low-powered sensors in mobile devices with a Convolutional Neural Network to help reduce battery usage. I developed an Android app in Java to collect the training and testing data, and implemented the CNN using Python and PyTorch.}

%----------------------------------------------------------------------------------------
%	WORK EXPERIENCE SECTION
%----------------------------------------------------------------------------------------

\section{Professional Experience}

%\subsection{Vocational}

\cventry{09/2020--Present}{Associate Solutions Architect}{\textsc{Amazon Web Services}}{Milan}{}{
    My main role as a Solution Architect in AWS involved talking directly to greenfield clients in the 
    retail space. I've presented AWS to multiple large Italian companies, and assisted them in developing 
    new applications on AWS or migrating parts of their infrastructure from on-premises to the cloud.
    \newline{}
    While working with AWS, the first 5 months were dedicated to studying the platform's services. I've personally 
    used multiple services like EC2, Lambda, DynamoDB, RedShift, CloudWatch, S3, SNS, Amplify to name a few. 
    During this period I also developed a couple of file hosting services fully on AWS to fill in some missing 
    gaps in the tools we had at the time.
}

%------------------------------------------------

\cventry{05/2019--08/2020}{Programmer}{\textsc{Reply S.p.A}}{Turin}{}{
    Worked and lead on multiple projects for external clients. The majority of the time I worked as 
    an iOS developer and maintainer, developing a couple of apps while also managing launches along 
    with my couterparts on Android. I also worked as a backend developer on other projects, mainly 
    creating backends in node.js
\newline{}\newline{}
Detailed projects:
\begin{itemize}
\item \textbf{Beretta Firearms}
\newline
    My work on the Shooting Data project involved learning Objective-C 
    and then taking over the role of primary iOS developer on the app, alongside
    the role as maintainer of a parser for data collection. As maintainer of the parser, I had
    to interact directly with the clients, use the Azure IoT Hub, and update the parser written
    in java.
\item \textbf{Vodafone}
\newline
    For this project, I maintained the main backend in node.js for the Smart@Home Vodafone Platform. 
    This required managing multiple databases for each development environment and general maintenance 
    of the backend for push notifications. The main technologies used were node.js, MySQL, and multiple 
    AWS technologies such as SQS, SNS, ECS, RDS, etc.
\item \textbf{Fondazione Torino Musei}
\newline
    On this project, I was given the role of developing the iOS app and designing the backend with 
    Firebase Cloud Firestore. I also had the opportunity to study and implement some Firestore Cloud 
    Functions to keep the data synchonized.
\item \textbf{Caleffi}
\newline
    I worked as an iOS developer on the project, which required designing multiple API 
    calls to our backend, and to develop an interaction flow between a gateway and iPhone 
    via bluetooth commands.
\end{itemize}}

%----------------------------------------------------------------------------------------
%	CERTIFICATIONS SECTION
%----------------------------------------------------------------------------------------

\section{Certifications}

\cvitem{2021}{AWS Certified Developer - Associate}
\cvitem{2020}{AWS Certified Solutions Architect - Associate}
\cvitem{2020}{AWS Certified Cloud Practitioner}

%----------------------------------------------------------------------------------------
%	COMPUTER SKILLS SECTION
%----------------------------------------------------------------------------------------

\section{Computer skills}

\cvitem{Basic}{\textsc{java}, Adobe Illustrator}
\cvitem{Intermediate}{\textsc{python}, \textsc{html}, \LaTeX, OpenOffice, Linux, Microsoft Windows}
\cvitem{Advanced}{Computer Hardware and Support}

%----------------------------------------------------------------------------------------
%	COMMUNICATION SKILLS SECTION
%----------------------------------------------------------------------------------------

\section{Communication Skills}

\cvitem{2010}{Oral Presentation at the California Business Conference}
\cvitem{2009}{Poster at the Annual Business Conference in Oregon}

%----------------------------------------------------------------------------------------
%	LANGUAGES SECTION
%----------------------------------------------------------------------------------------

\section{Languages}

\cvitemwithcomment{English}{Mothertongue}{C2}
\cvitemwithcomment{Italian}{Mothertongue}{C2}
\cvitemwithcomment{French}{Basic}{A2}

%----------------------------------------------------------------------------------------
%	INTERESTS SECTION
%----------------------------------------------------------------------------------------

\section{Interests}

\renewcommand{\listitemsymbol}{-~} % Changes the symbol used for lists

\cvlistdoubleitem{Piano}{Chess}
\cvlistdoubleitem{Cooking}{Dancing}
\cvlistitem{Running}

%----------------------------------------------------------------------------------------

\end{document}